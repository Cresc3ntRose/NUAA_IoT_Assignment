\documentclass[12pt, a4paper, oneside]{ctexart}
\usepackage{amsmath, amsthm, amssymb, bm, color, enumerate, framed, graphicx, longtable, mathrsfs, subfigure, tikz}
\usepackage{geometry}
\geometry{left=2.54cm,right=2.54cm,top=3.18cm,bottom=3.18cm}

% 超链接设置
\usepackage{hyperref}
\hypersetup{  
    colorlinks = true,    % 更改链接颜色  
    linkcolor = purple,      % 链接颜色  
    urlcolor = blue,        % URL 颜色  
    citecolor = green,     % 引用颜色  
    % underline = true,  
    linkbordercolor = red,
}
% \hypersetup{colorlinks=true,linkcolor=black}

% 附录
\newcommand{\append}{\textsc{Appendix}}

%代码包设置
\usepackage{listings}
\lstset{
    basicstyle          =   \sffamily,          % 基本代码风格
    keywordstyle        =   \bfseries,          % 关键字风格
    commentstyle        =   \rmfamily\itshape,  % 注释的风格,斜体
    stringstyle         =   \ttfamily,  % 字符串风格
    flexiblecolumns,                % 别问为什么,加上这个
    numbers             =   left,   % 行号的位置在左边
    showspaces          =   false,  % 是否显示空格,显示了有点乱,所以不现实了
    numberstyle         =   \zihao{-5}\ttfamily,    % 行号的样式,小五号,tt等宽字体
    showstringspaces    =   false,
    captionpos          =   t,      % 这段代码的名字所呈现的位置,t指的是top上面
    frame               =   lrtb,   % 显示边框
}
% \lstdefinestyle{C}{
%     language        =   C, % 语言选C
%     basicstyle      =   \zihao{-5}\ttfamily,
%     numberstyle     =   \zihao{-5}\ttfamily,
%     keywordstyle    =   \color{blue},
%     keywordstyle    =   [2] \color{teal},
%     stringstyle     =   \color{magenta},
%     commentstyle    =   \color{red}\ttfamily,
%     breaklines      =   true,   % 自动换行,建议不要写太长的行
%     columns         =   fixed,  % 如果不加这一句,字间距就不固定,很丑,必须加
%     basewidth       =   0.5em,
% }
\lstdefinestyle{Python}{
    language        =   Python, % 语言选Python
    basicstyle      =   \zihao{-5}\ttfamily,
    numberstyle     =   \zihao{-5}\ttfamily,
    keywordstyle    =   \color{blue},
    keywordstyle    =   [2] \color{teal},
    stringstyle     =   \color{magenta},
    commentstyle    =   \color{red}\ttfamily,
    breaklines      =   true,   % 自动换行,建议不要写太长的行
    columns         =   fixed,  % 如果不加这一句,字间距就不固定,很丑,必须加
    basewidth       =   0.5em,
}

% 设置思源宋体为中文字体
\setCJKsansfont{Source Han Serif SC}

% 设置思源黑体为中文字体
\setCJKsansfont{Source Han Sans SC}

% 设置arev为英文字体
\setsansfont{Arev Sans}

% 字体颜色
\def\red{\color{red}}
\def\blue{\color{blue}}
\def\black{\color{black}}
\def\green{\color{green}}

\title{\textbf{数据库原理课程作业}}
\author{162140222黄钰轩}
\date{\today}
\linespread{1.25}

\definecolor{shadecolor}{RGB}{241, 241, 255}
\newcounter{problemname}
\newenvironment{problem}{\begin{shaded}\stepcounter{problemname}\par\noindent\textbf{题目\arabic{problemname}. }}{\end{shaded}\par}
\newenvironment{solution}{\par\noindent\textbf{解答. }}{\par}
\newenvironment{note}{\par\noindent\textbf{题目\arabic{problemname}的注记. }}{\par}
%\renewcommand{\proofname}{\bf 证明}

\begin{document}

\begin{figure}[t]
    \centering
    \includegraphics[width=13cm]{logo.png}
\end{figure}

\maketitle

\begin{problem}
    设有关系模式$R<U, F>$, 其中$U = \{ A, B, C, D, E \}, F = \{ A \rightarrow B, BC \rightarrow E, ED \rightarrow AB \}$.
    \begin{enumerate}[(1)]
        \item 计算$A_F^+$、$(AB)_F^+$、$(ABC)_F^+$、$(BCD)_F^+$.
        \item 求$R$的所有候选码,并说明理由.
        \item $R$最高满足第几范式?为什么?
    \end{enumerate}
\end{problem}

\begin{solution}
    \begin{enumerate}[(1)]
        \item 
            \begin{itemize}
                \item $A_F^+$
                \begin{enumerate}[$1^\circ$]
                    \item 令$A^{(0)} = A$
                    \item 在$F$中找到左边是$A$子集的函数依赖,有$A \rightarrow B$, \\
                    则$A^{(1)} = A^{(0)}\bigcup B = AB$
                    \item 虽然$A^{(1)} \not= A^{(0)}$,但是$F$中未用过的函数依赖的左边属性集中已经没有$A^{(1)}$自己了,因此可以退出循环
                    \item 输出$A_F^+ = AB$
                \end{enumerate}
                \newpage
                \item $(AB)_F^+$
                \begin{enumerate}[$1^\circ$]
                    \item 令$(AB)^{(0)} = AB$
                    \item 在$F$中找到左边是$AB$子集的函数依赖,有$A \rightarrow B$, \\
                    则$(AB)^{(1)} = (AB)^{(0)}\bigcup B = AB$
                    \item $(AB)^{(1)} = (AB)^{(0)}$,退出循环
                    \item 输出$(AB)_F^+ = AB$
                \end{enumerate}
                \item $(ABC)_F^+$
                \begin{enumerate}[$1^\circ$]
                    \item 令$(ABC)^{(0)} = ABC$
                    \item 在$F$中找到左边是$ABC$子集的函数依赖,有$A \rightarrow B, BC \rightarrow E$, \\
                    则$(ABC)^{(1)} = (ABC)^{(0)}\bigcup BE = ABCE$
                    \item 虽然$(ABC)^{(1)} \not= (ABC)^{(0)}$,但是$F$中未用过的函数依赖的左边属性集中已经没有$(ABC)^{(1)}$自己了,因此可以退出循环
                    \item 输出$(ABC)_F^+ = ABCE$
                \end{enumerate}
                \item $(BCD)_F^+$
                \begin{enumerate}[$1^\circ$]
                    \item 令$(BCD)^{(0)} = BCD$
                    \item 在$F$中找到左边是$BCD$子集的函数依赖,有$BC \rightarrow E$, \\
                    则$(BCD)^{(1)} = (BCD)^{(0)}\bigcup E = BCDE$
                    \item 在$F$中找到左边是$BCDE$子集的函数依赖,有$ED \rightarrow AB$, \\
                    则$(BCD)^{(2)} = (BCD)^{(1)}\bigcup AB = ABCDE$
                    \item $(BCD)^{(2)} = U$,\\
                    则$(BCD)^{(2)}$就是$(BCD)_F^+$
                    \item 输出$(BCD)_F^+ = ABCDE$
                \end{enumerate}
            \end{itemize}
        \item 
            \begin{enumerate}[$1^\circ$]
                \item 计算$L$类、$N$类、$LR$类属性. $L$类属性有$CD$, $N$类属性为空,\\
                $LR$类属性有$ABE$. 令$X$为$L$类属性的集合,$Y$为$LR$类属性的集合
                \item 计算属性集$X$关于$U$上的函数依赖集$F$的闭包$X_F^+$,有$X_F^+ = CD \not= U$
                \item 遍历$Y$中的单一属性,并与$X$构成属性组,后计算闭包. 有:
                    \begin{itemize}
                        \item $(XA)_F^+ = (ACD)_F^+ = ABCDE$,因此$ACD$是候选码
                        \item $(XB)_F^+ = (BCD)_F^+ = ABCDE$,因此$BCD$是候选码
                        \item $(XE)_F^+ = (CDE)_F^+ = ABCDE$,因此$CDE$是候选码
                    \end{itemize}
            \end{enumerate}
            综上,关系模式$R<U, F>$的所有候选码为$ACD$、$BCD$、$CDE$.
        \newpage
        \item 
            \begin{enumerate}[$1^\circ$]
                \item 不存在非主属性对码的部分函数依赖,所以$R$至少满足 2NF.
                \item 不存在非主属性对码的传递函数依赖,所以$R$至少满足 3NF.
                \item 存在主属性对码的部分函数依赖,所以$R$不满足 BCNF.
            \end{enumerate}

            综上,$R$最高满足 3NF.
        \end{enumerate}
\end{solution}

\begin{problem}
    设有关系模式$R<U, F>$, 其中$U = \{ A, B, C, D, E, F \}$, \\$F = \{ A \rightarrow BC, CD \rightarrow E, B \rightarrow D, E \rightarrow A, CD \rightarrow A, A \rightarrow D \}$. 求: 
    \begin{enumerate}[(1)]
        \item 最小依赖集.
        \item 所有的候选码.
    \end{enumerate}
\end{problem}

\begin{solution}
    \begin{enumerate}[(1)]
        \item 
            \begin{enumerate}[$1^\circ$]
                \item $F = \{ A \rightarrow B, A \rightarrow C, CD \rightarrow E, B \rightarrow D, E \rightarrow A, CD \rightarrow A, A \rightarrow D \}$
                \item 
                    \begin{itemize}
                        \item 考察$A \rightarrow B, G = \{ A \rightarrow C, CD \rightarrow E, B \rightarrow D, E \rightarrow A, CD \rightarrow A, A \rightarrow D \},$ \\
                            $A_G^+ = ACDE, A \rightarrow B$ 保留
                        \item 考察$A \rightarrow C, G = \{ A \rightarrow B, CD \rightarrow E, B \rightarrow D, E \rightarrow A, CD \rightarrow A, A \rightarrow D \},$ \\
                        $A_G^+ = ABD, A \rightarrow C$ 保留
                        \item 考察$CD \rightarrow E, G = \{ A \rightarrow B, A \rightarrow C, B \rightarrow D, E \rightarrow A, CD \rightarrow A, A \rightarrow D \},$ \\
                        $(CD)_G^+ = ABCD, CD \rightarrow E$ 保留
                        \item 考察$B \rightarrow D, G = \{ A \rightarrow B, A \rightarrow C, CD \rightarrow E, E \rightarrow A, CD \rightarrow A, A \rightarrow D \},$ \\
                        $B_G^+ = B, B \rightarrow D$ 保留
                        \item 考察$E \rightarrow A, G = \{ A \rightarrow B, A \rightarrow C, CD \rightarrow E, B \rightarrow D, CD \rightarrow A, A \rightarrow D \},$ \\
                        $E_G^+ = E, E \rightarrow A$ 保留
                        \item 考察$CD \rightarrow A, G = \{ A \rightarrow B, A \rightarrow C, CD \rightarrow E, B \rightarrow D, E \rightarrow A, A \rightarrow D \},$ \\
                        $(CD)_G^+ = ABCDE, CD \rightarrow A$ 是多余的
                        \item 考察$A \rightarrow D, G = \{ A \rightarrow B, A \rightarrow C, CD \rightarrow E, B \rightarrow D, E \rightarrow A \},$ \\
                        $A_G^+ = ABCDE, A \rightarrow D$ 是多余的
                    \end{itemize}
                    于是$H = \{ A \rightarrow B, A \rightarrow C, CD \rightarrow E, B \rightarrow D, E \rightarrow A \}$
                \item 考察左部多余的属性
                    \begin{itemize}
                        \item 考察$CD \rightarrow E$
                            \begin{itemize}
                                \item $C_H^+ = C$, 不包含$E$, $C$ 保留
                                \item $D_H^+ = D$, 不包含$E$, $D$ 保留
                            \end{itemize}
                    \end{itemize}
            \end{enumerate}
            于是$$F_{min} = \{ A \rightarrow B, A \rightarrow C, CD \rightarrow E, B \rightarrow D, E \rightarrow A \}$$
        \item $F = \{ A \rightarrow B, A \rightarrow C, CD \rightarrow E, B \rightarrow D, E \rightarrow A \}$
            \begin{enumerate}[$1^\circ$]
                \item 计算$L$类、$N$类、$LR$类属性. $L$类属性有$B$, $N$类属性有$F$,\\
                $LR$类属性有$ACDE$. 令$X$为$L$类属性集与$N$类属性集的并集,$Y$为$LR$类属性的集合
                \item 计算属性集$X$关于$U$上的函数依赖集$F$的闭包$X_F^+$,有$X_F^+ = BDF \not= U$
                \item 遍历$Y$中的单一属性,并与$X$构成属性组,后计算闭包. 有:
                    \begin{itemize}
                        \item $(XA)_F^+ = (ABF)_F^+ = ABCDEF$,因此$ABF$是候选码
                        \item $(XC)_F^+ = (BCF)_F^+ = BCF \not= U$,因此$BCF$不是候选码
                        \item $(XD)_F^+ = (BDF)_F^+ = BDF \not= U$,因此$BDF$不是候选码
                        \item $(XE)_F^+ = (BEF)_F^+ = ABCDEF$,因此$BEF$是候选码
                    \end{itemize}
                \item 此时 $Y = \{ C, D \}$, 遍历$Y$中的任意两个属性,并与$X$构成属性组,后计算闭包. 有:
                    \begin{itemize}
                        \item $Z = CD, (XZ)_F^+ = (BCDF)_F^+ = ABCDEF$,因此$BCDF$是候选码
                    \end{itemize}
            \end{enumerate}
            综上,关系模式$R<U, F>$的所有候选码为$ABF$、$BEF$、$BCDF$.
    \end{enumerate}
\end{solution}

\begin{problem}
    给定关系$Q(A, B, C, D, E)$如\hyperref[Q]{表 1 }所示,试给出至少2个非对称多值依赖.
\end{problem}

\begin{longtable}{|c|c|c|c|c|}
    \hline
    \bf A & \bf B & \bf C & \bf D & \bf E \\
    \hline
    $a$ & $b$ & $c$ & $d$ & $e$ \\
    \hline
    $a$ & $b'$ & $c'$ & $d'$ & $e'$ \\
    \hline
    $a$ & $b$ & $c$ & $d'$ & $e'$ \\
    \hline
    $a$ & $b'$ & $c'$ & $d$ & $e$ \\
    \hline
    $a$ & $b$ & $c$ & $d$ & $e'$ \\
    \hline
    $a$ & $b$ & $c$ & $d'$ & $e$ \\
    \hline
    $a$ & $b'$ & $c'$ & $d'$ & $e$ \\
    \hline
    $a$ & $b'$ & $c'$ & $d$ & $e'$ \\
    \hline
    \caption{关系$Q$}
    \label{Q}
\end{longtable}
\newpage
\begin{solution}
    \begin{itemize}
        \item 
            令$$X = \{ A, B \}, Y = \{ C \},$$
            于是$$Z = U - X - Y = \{ D, E \}.$$
            对于任意一对$(x, z)$, $y$的值仅由$x$决定而与$z$无关,
            于是存在多值依赖$$X\rightarrow\rightarrow Y,$$
            此时$X\rightarrow\rightarrow Z$并不成立, 所以$X\rightarrow\rightarrow Y$是一个非对称多值依赖.
        \item 
            令$$X = \{ A, C \}, Y = \{ B \},$$
            于是$$Z = U - X - Y = \{ D, E \}.$$
            对于任意一对$(x, z)$, $y$的值仅由$x$决定而与$z$无关,
            于是存在多值依赖$$X\rightarrow\rightarrow Y,$$
            此时$X\rightarrow\rightarrow Z$并不成立, 所以$X\rightarrow\rightarrow Y$是一个非对称多值依赖.
    \end{itemize}
\end{solution}

\begin{problem}
    试由 Armstrong 公理系统推导出下面的推理规则:
    
    若$A \rightarrow B, BC \rightarrow D$, 则有 $AC \rightarrow D$.
\end{problem}

\begin{solution}
    \begin{enumerate}[$1^\circ$]
        \item 由题中给出$A \rightarrow B$成立,\\根据增广律,可以推出$AC\rightarrow BC$成立.
        \item 由题中给出$BC \rightarrow D$成立,\\根据传递律,可以推出$AC\rightarrow D$成立,\\题目得证.
    \end{enumerate}
\end{solution}

\begin{note}
    本题事实上完成了\textbf{伪传递规则}的推导.
\end{note}

\newpage
\begin{problem}
    设有如\hyperref[S]{表 2 }所示的学生关系$S$.
    \begin{enumerate}[(1)]
        \item 试问$S$是否属于 3NF, 请做出判断并给出理由.
        \item 若不是 3NF, 它属于第几范式.
        \item 若不是 3NF, 请将其规范化为 3NF.
    \end{enumerate}
\end{problem}

\begin{longtable}{|c|c|c|c|c|c|}
    \hline
    \bf 学号 & \bf 学生名 & \bf 年龄 & \bf 性别 & \bf 系号 & \bf 系名 \\
    \hline
    10001 & 王婧 & 18 & 女 & 1 & 通信工程 \\
    20001 & 张露 & 19 & 女 & 2 & 电子工程 \\
    20002 & 黎明远 & 20 & 男 & 2 & 电子工程 \\
    30001 & 王烨 & 21 & 男 & 3 & 计算机 \\
    30004 & 张露 & 20 & 女 & 3 & 计算机 \\
    30005 & 潘建 & 19 & 男 & 3 & 计算机 \\
    \hline
    \caption{关系$S$}
    \label{S}
\end{longtable}

\begin{solution}
    令$\text{学号} = A, \text{学生名} = B, \text{年龄} = C, \text{性别} = D, \text{系号} = E, \text{系名} = F,$
    则有关系模式$S<U,F>$, 其中
    $$
        \begin{aligned}
            & U = \{A, B, C, D, E, F\} \\
            & \begin{aligned}
                F = \{ &A\rightarrow B, \\
                & A\rightarrow C, \\
                & A\rightarrow D, \\
                & A\rightarrow E, \\
                & A\rightarrow F, \\
                & E\rightarrow F\}
              \end{aligned}
        \end{aligned}
    $$
    接下来求解$S$的所有候选码:
    \begin{enumerate}[$1^\circ$]
        \item 计算$L$类、$N$类、$LR$类属性. $L$类属性有$A$, $N$类属性为空,\\
        $LR$类属性有$E$. 令$X$为$L$类属性的集合,$Y$为$LR$类属性的集合
        \item 计算属性集$X$关于$U$上的函数依赖集$F$的闭包$X_F^+$,有$X_F^+ = ABCDE = U$
        \item 因为$X = \{ A \}$, 因此$A$是唯一候选码
    \end{enumerate}
    综上,关系模式$S<U, F>$的所有候选码为$A$.
    \begin{enumerate}[(1)]
        \item 
            有$$A\rightarrow E, E\rightarrow F,$$
            且$$E\not\rightarrow A.$$
            存在主属性对码的传递函数依赖,所以$S$不属于3NF.
        \item 
            \begin{enumerate}[$1^\circ$]
                \item 不存在非主属性对码的部分函数依赖,所以$S$最低满足2NF.
                \item 存在非主属性对码的传递函数依赖,所以$S$不满足3NF.
            \end{enumerate}
            综上,$S$ 最高满足 2NF.
        \item 
            可以将关系模式$S<U, F>$拆分为$S_1<U_1, F_1>$、$S_2<U_2, F_2>$与$S_3<U_3, F_3>$, 其中
            $$
                \begin{aligned}
                    & U_1 = \{A, B, C, D\} \\
                    & \begin{aligned}
                        F_1 = \{ &A\rightarrow B, \\
                        & A\rightarrow C, \\
                        & A\rightarrow D\}
                    \end{aligned} \\
                    & U_2 = \{A, E\} \\
                    & F_2 = \{A \rightarrow E\} \\
                    & U_3 = \{E, F\} \\
                    & F_3 = \{E \rightarrow F\} \\
                \end{aligned}
            $$
            这样,便消除了主属性对码的部分和传递函数依赖.
    \end{enumerate}
\end{solution}

\end{document}